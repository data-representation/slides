%!TEX root = ../slides.tex
\section{REST}


\begin{frame}{REST}
  \begin{itemize}
    \item REST stands for Representational State Transfer.
    \item REST is an architecture describing how we might use HTTP.
    \item RESTful APIs make use of more HTTP methods than just GET and POST.
    \item Most HTTP APIs are not RESTful.
    \item RESTful APIs adhere to a few loosely defined constraints.
    \item Two of those constraints are that the API is stateless and cacheable.
  \end{itemize}
  \citeurl{drdobbs.com/web-development/restful-web-services-a-tutorial/240169069}
\end{frame}


\begin{frame}{Typical example}
  Suppose we have a system for storing and retrieving emails.
  \begin{table}
    \begin{tabular}{r@{\hspace{0.5cm}}|l@{\hspace{0.5cm}}|l}
      Method & URL & Description \\
      \hline
      GET    & /emails   & list all emails \\
      POST   & /email    & store new email \\
      GET    & /email/32 & retrieve email with id 32 \\
      PUT    & /email/32 & update email with id 32 \\
      DELETE & /email/32 & delete email with id 32
    \end{tabular}
  \end{table}
  \citeurl{bitworking.org/news/201/restify-daytrader}
\end{frame}


\begin{frame}{Stateless}
  \begin{itemize}
    \item Statelessness is a REST constraint.
    \item HTTP uses the client-server model.
    \item The server should treat each request as a single, independent transaction.
    \item No client state should be stored on the server.
    \item Each request must contain all of the information to perform the request. 
  \end{itemize}
\end{frame}


\begin{frame}{Cacheable}
  \begin{itemize}
    \item REST APIs should provide responses that are cacheable.
    \item Intermediaries between the client and server should be able to cache responses.
    \item This should be transparent to the client.
    \item Cacheability increases response time.
    \item Browsers usually cache resources, in case they are requested again.
    \item There is usually a time limit on cached resources.
  \end{itemize}
\end{frame}