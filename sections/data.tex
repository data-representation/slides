%!TEX root = ../slides.tex
\section{Data}


\begin{frame}{JSON}
  \begin{description}
    \item[JavaScript] A scripting/programming language.
    \vspace{0.25cm}
    \item[Object] Groups of name--value pairs.
    \vspace{0.25cm}
    \item[Notation] Set of rules for representing objects.
  \end{description}
  \begin{itemize}
    \item JSON is just text --- text that conforms to a syntax.
    \item JSON is heavily influenced by JavaScript, but it is used in with all languages.
    \item JSON's primary purpose is to represent information in text form.
    \item JSON is popular because it is easy to send over HTTP and parse in JavaScript.
  \end{itemize}
\end{frame}

\begin{frame}{Sending JSON}
  \tikzstyle{block} = [rectangle, fill=azure(colorwheel), text width=4.5em, text centered, minimum height=4em]
  \tikzstyle{line} = [draw, -latex']
  \tikzstyle{cloud} = [ellipse, fill=wildwatermelon, text width=5em, text centered]
  \tikzstyle{startstop} = [rectangle, rounded corners, text width=5em, minimum height=4em, text centered, fill=tiffanyblue]
  
  \begin{adjustbox}{max totalsize={.9\textwidth}{.6\textheight},center}
  
  \tikzstyle{rect} = [rectangle,fill=gmitblue,text width=4.5em,text centered,minimum height=4em,rounded corners,text=white]
  \tikzstyle{line} = [draw,->,very thick]
  \tikzstyle{oval} = [ellipse,fill=gmitred,text width=5em,text centered,text=white]
    
    \begin{tikzpicture}[node distance=4cm]
    \node [oval] (object1) {Object Instance};
    \node [rect, below of=object1] (json1) {JSON};
    \node [rect, right of=json1, node distance = 6cm] (json2) {JSON};
    \node [oval, above of=json2] (object2) {Object Instance};
    % Draw edges
    \path [line] (object1) -- node[style={rectangle,fill=white,draw}]{stringify} (json1);
    \path [line] (json1) -- node[style={rectangle,fill=white},draw ]{HTTP} (json2);
    \path [line] (json2) -- node[style={rectangle,fill=white,draw}]{parse} (object2);
    % Draw Memory
    \draw [color=gray, dashed](-2,-1.5) rectangle (2,1.25);
    \node at (-1.4,-1.35) [] {\tiny Machine 1};
    \draw [color=gray, dashed](4,-1.5) rectangle (8,1.25);
    \node at (4.6,-1.35) [] {\tiny Machine 2};
    \end{tikzpicture}
  \end{adjustbox}
\end{frame}

\begin{frame}[fragile]{JSON Example}
  \begin{minted}{json}
{
  "employees": [
    {"firstName":"John", "lastName":"Doe"},
    {"firstName":"Anna", "lastName":"Smith"},
    {"firstName":"Peter", "lastName":"Jones"}
  ]
}
  \end{minted}
\end{frame}


\begin{frame}[fragile]{Using JSON in JavaScript}
  \begin{minted}{javascript}
// Turning text into a JavaScript object.
var obj = JSON.parse(text);
// obj is an obect.

// Turning a JavaScript object into text.
var text = JSON.stringify(obj);
// text is a string.
  \end{minted}
\end{frame}

\begin{frame}{JSON Syntax}
  \begin{itemize}
    \item Name/Value pairs separated by a colon. \\
    \hspace{0.5cm} \mintinline{json}{"name": "Ian"}
    \item Objects identified by curly braces. \\
    \hspace{0.5cm} \mintinline{json}{{}}
    \item Lists identified by square brackets. \\
    \hspace{0.5cm} \mintinline{json}{[]}
    \item All strings (and names) use double quotes (not single). \\
    \hspace{0.5cm} \mintinline{json}{"Ian"}
  \end{itemize}
\end{frame}

\begin{frame}{JSON Types}
  \begin{itemize}
    \item Numbers \\
    \hspace{0.5cm} \mintinline{json}{123.456}
    \item Strings \\
    \hspace{0.5cm} \mintinline{json}{"Hello, world!"}
    \item Boolean \\
    \hspace{0.5cm} \mintinline{json}{true"}
    \item Arrays\\
    \hspace{0.5cm} \mintinline{json}{[1,2,3]}
    \item Objects\\
    \hspace{0.5cm} \mintinline{json}{{"name": "Ian"}}
    \item null \\
    \hspace{0.5cm} \mintinline{json}{null}
  \end{itemize}
\end{frame}



\begin{frame}{eXtensible Markup Language}
  \begin{description}
    \item[Extensible] Designed to accommodate change.
    \vspace{0.25cm}
    \item[Markup] Annotates text.
    \vspace{0.25cm}
    \item[Language] Set of rules for communication.
  \end{description}
\end{frame}


\begin{frame}{About XML}
  \begin{itemize}
    \item XML is an alternative to JSON.
    \item XML looks like HTML, but it is different.
    \item XML's purpose is to represent information in text form.
    \item There are no pre-defined tag names -- you make them up yourself.
    \item XML has a tree-like syntax.
    \item The Document Object Model (DOM) can be applied to XML.
  \end{itemize}
\end{frame}


\begin{frame}[fragile]{XML Example}
  \begin{minted}{xml}
<?xml version="1.0" encoding="UTF-8"?>
<book isbn-13="978-0131774292" isbn-10="0131774298">
  <title>Expert C Programming: Deep C Secrets</title>
  <publisher>Prentice Hall</publisher>
  <author>Peter van der Linden</author>
</book>
  \end{minted}
\end{frame}

\begin{frame}{XML Syntax}
  \begin{description}
    \item[Declaration] XML documents should have a single line at the start stating that it's XML, the version of XML it is, and an encoding.
    \item[Elements] XML is structured as elements, which are enclosed in angle brackets.
    \item[Root element] XML must have a single root element that wraps all others.
    \item[Attbirutes] Elements can have attributes, which are name--value pairs within the angle brackets. A given attribute name can only be specified once per element.
    \item[Entity references] Certain characters must be escaped with entity references, e.g.\ \&lt; for $\langle$.
    \item[Case sensitive] Everything in XML is case sensitive.
  \end{description}
\end{frame}

\begin{frame}[fragile]{XML Syntax Example}
  \begin{minted}{xml}
  <?xml version="1.0" encoding="UTF-8"?>
  <parent-element attribute-name="attribute-value">
    <child name="value">Text</child-element>
    <child name="value">Text</child-element>
    <child name="value">Text</child-element>
    <lone-warrior />
  </parent-element>
  \end{minted}
\end{frame}

\begin{frame}{Document Object Model}
  \begin{itemize}
    \item The Document Object Model (DOM) is a programming interface for HTML and XML documents.
    \item It provides a model of the document as a structured group of nodes that have properties and methods.
    \item The DOM connects web pages to scripts or programming languages.
    \item You can use document.createElement, document.createTextNode and document.element.appendChild to add to the DOM.
    \item You can use document.getElementById to access elements of the DOM.
  \end{itemize}
  \citeurl{developer.mozilla.org/en-US/docs/Web/API/Document\_Object\_Model/Introduction}
\end{frame}


\begin{frame}{Asynchronous JavaScript and XML}
  AJAX stands for Asynchronous JavaScript and XML.
  \vspace{0.5cm}
  \begin{description}
    \item[Asynchronous] In the background, and without a page refresh.
    \vspace{0.25cm}
    \item[JavaScript] Programming language for the web.
    \vspace{0.25cm}
    \item[XML] eXtensible Markup Language.
  \end{description}
\end{frame}


\begin{frame}{About AJAX}
  \begin{itemize}
    \item AJAX allows us to make a HTTP request from JavaScript without a page refresh.
    \item AJAX also allows us to receive the response from that request and deal with it.
    \item Despite the name, we don't have to use XML -- we can use JSON or anything else.
    \item This happens asynchronously, so that the rest of our code be run while waiting for a slower piece of code to complete.
    \item HTTP requests are usually relatively slow.
    \item We use a callback function, which is called when the HTTP transaction is complete.
  \end{itemize}
\end{frame}


\begin{frame}[fragile]{AJAX Example}
  \begin{minted}{javascript}
var xmlhttp = new XMLHttpRequest();

xmlhttp.onreadystatechange = function() {
  if (xmlhttp.readyState == 4) {
    var mydiv = document.getElementById("mydivid");
    mydiv.innerHTML = xmlhttp.responseText;
  }
};

xmlhttp.open("GET", "https://goo.gl/2GCplC");
xmlhttp.send();
  \end{minted}
\end{frame}


\begin{frame}{AJAX Example Explained}
  \begin{itemize}
    \item XMLHttpRequest is a built-in class that provides AJAX functionality in JavaScript.
    \item httpRequest.onreadystatechange should be set to a function to run every time something happens in our HTTP call.
    \item httpRequest.open is called to initialize the request.
    \item httpRequest.send is used to send the request to the server.
    \item XMLHttpRequest.readyState changes when the state of the AJAX call changes. This triggers a call to httpRequest.onreadystatechange.
  \end{itemize}
\end{frame}


\begin{frame}[fragile]{Using jQuery}
  \begin{minted}{html}
<script src="jquery.min.js"></script>
  \end{minted}
  \vspace{1cm}
  \begin{minted}{javascript}
$.get("https://goo.gl/2GCplC", function(data) {
  $("#mydivid").html(data);
});
  \end{minted}
\end{frame}