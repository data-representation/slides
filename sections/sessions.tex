%!TEX root = ../slides.tex
\section{Sessions}

\begin{frame}{Sessions}
  \begin{description}
    \item[HTTP] treats each request--response individually.
		\item[How] can we let users identify themselves to the server as the same user who made a previous request?
		\item[Needed] to enable such things as shopping carts.
		\item[Sessions] provide a mechanism for this.
		\item[Servers] can provide a unique key in a response, which can be used in the next request.
		\item[Amazon] were one of the first companies to design this functionality.
  \end{description}
\end{frame}



\begin{frame}{Security}
  \begin{itemize}
    \item HTTP is not encrypted.
    \item HTTPS is a protocol based on HTTP, but it provides security.
    \item GET and POST are by far the most commonly used HTTP methods (by web developers).
    \item Data sent by GET and POST will be encrypted over HTTPS.
    \item However, it's generally accepted that POST is more secure for sending sensitive data.
    \item This is because browsers will typically cache and servers will typically log URLS, with the data encoded in them.
  \end{itemize}
\end{frame}


\begin{frame}{HTTP APIs}
	\begin{itemize}
		\item Facebook, Google, Reddit and others often provide programmable interfaces to their services.
		\item This lets other application developers use the services programmatically.
		\item For instance, Reddit allows developers to create mobile apps for viewing and making submissions to reddit.
		\item HTTP is often the mechanism used for this purpose.
		\item Access is provided through a set of URLs, across a variety of HTTP methods.
		\item The APIs often require JSON in HTTP request bodies and often return the query results as JSON.
	\end{itemize}
\end{frame}


\begin{frame}{NoSQL}
	\begin{itemize}
		\item NoSQL is the umbrella term for databases that do not conform to the relational, SQL-style model.
		\item Relational databases are good for some types of data.
		\item However, they have some issues.
		\item SQL queries can result in costly joins.
		\item Tables can be sparsely populated.
		\item Two common NoSQL database types are Document-oriented and Graph.
	\end{itemize}
\end{frame}


\begin{frame}{CouchDB}
	\begin{itemize}
		\item CouchDB is a document-oriented database.
		\item Documents are represented in CouchDB as JSON objects.
		\item Each document has its own id and revision, indicated by properties \mintinline{js}{_id} and \mintinline{js}{_rev} in the JSON document.
		\item Updating a document leaves its \mintinline{js}{_id} intact, but updates its \mintinline{js}{_rev}.
		\item Different documents can have different properties -- there is no schema.
		\item The main interface with CouchDB, for storage and retrieval is a HTTP API.
		\item CouchDB uses HTTP methods such as \mintinline{http}{GET}, \mintinline{http}{POST}, \mintinline{http}{PUT} and \mintinline{http}{DELETE} to retrieve, add, update and delete documents.
	\end{itemize}
\end{frame}


\begin{frame}{Futon}
	\begin{itemize}
		\item CouchDB has an in-built admin interface.
		\item It's called Futon.
		\item You access it through the \mintinline{html}{/_utils} path.
		\item You can create and delete databases.
		\item You can also create, update and delete documents.
	\end{itemize}
\end{frame}


\begin{frame}{MapReduce}
	\begin{itemize}
		\item MapReduce is a way of programming.
		\item It is a model for performing specific types of problems that are common in programming.
		\item MapReduce promotes algorithms that have an initially embarrassingly parallel part, and a subsequent consolidation part.
		\item The former is the Map part, and the latter is the Reduce part.
		\item MapReduce isn't necessarily anything new, the ideas have existed for a long time.
		\item The formalisation of those ideas and their implementation in systems such as Hadoop is useful.
	\end{itemize}
\citeurl{joelonsoftware.com/items/2006/08/01.html}
\end{frame}


\begin{frame}[fragile]{Map}
Map takes a function and a list, and applies the function to every element of the list.
  \begin{minted}{javascript}
function map(fn, a) {
  r = [];
  for (i = 0; i < a.length; i++)
  	r[i] = fn(a[i]);
	return r;
}
	\end{minted}
	\citeurl{joelonsoftware.com/items/2006/08/01.html}
\end{frame}


\begin{frame}[fragile]{Reduce}
Reduce takes the output of Map, and accumulates the elements in some way.
  \begin{minted}{javascript}
function reduce(fn, a, init) {
  var s = init;
  for (i = 0; i < a.length; i++)
      s = fn(s, a[i]);
  return s;
}
	\end{minted}
	\citeurl{joelonsoftware.com/items/2006/08/01.html}
\end{frame}


\begin{frame}[fragile]{Map Reduce in CouchDB}
Reduce takes the output of Map, and accumulates the elements in some way.
  \begin{minted}{javascript}
function(doc) {
  if(doc.date && doc.title) {
    emit(doc.date, doc.title);
  }
}
	\end{minted}
	
  \begin{minted}{javascript}
function(keys, values, rereduce) {
  if (rereduce)
    return sum(values);
	else
    return values.length;
}
	\end{minted}
	
	\citeurl{http://guide.couchdb.org/draft/views.html}

\end{frame}


\begin{frame}{Redis}
  \begin{itemize}
    \item
  \end{itemize}
\end{frame}